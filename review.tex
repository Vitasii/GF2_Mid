\documentclass[twocolumn]{ctexart} % 使用ctex提供的文档类
%\usepackage[UTF8]{ctex} % 加载ctex宏包以支持中文

\usepackage{geometry}
\geometry{a4paper,left=1cm,right=1cm,top=2cm,bottom=2cm}
\usepackage{graphicx} %插入图片的宏包
\usepackage{float} %设置图片浮动位置的宏包
\usepackage{subfig}
\usepackage{cite}
\usepackage[colorlinks=true, urlcolor=blue, linkcolor=blue, citecolor=blue]{hyperref}
\usepackage{amsmath,amsfonts,amsthm,amssymb}
\numberwithin{equation}{subsection}
%\CTEXsetup[format={\Large\bfseries}]{section}
\usepackage{CJKutf8}
\usepackage{caption}


\title{General Physics 2 Summary}
\author{Sunflower027}
\date{2025.11}

\begin{document}
\maketitle

\section{矢量场}
\paragraph{向量运算}
\begin{align*}
	&\nabla (\varphi \psi) = \psi \nabla \varphi + \varphi \nabla \psi \\&
	\nabla \cdot (\varphi \vec{g}) = \nabla \varphi \cdot \vec{g} + \varphi \nabla \cdot \vec{g} \\&
	\nabla \times (\varphi \vec{g}) = \nabla \varphi \times \vec{g} + \varphi \nabla \times \vec{g} \\&
	\nabla \cdot (\vec{g} \times \vec{f}) = (\nabla \times \vec{g}) \cdot \vec{f} - \vec{g} \cdot (\nabla \times \vec{f}) \\&
	\nabla \times (\vec{g} \times \vec{f}) = (\vec{f} \cdot \nabla) \vec{g} - (\nabla \cdot \vec{g}) \vec{f} + (\nabla \cdot \vec{f}) \vec{g} - (\vec{g} \cdot \nabla) \vec{f} \\&
	\nabla \times (\nabla \times \vec{h}) = \nabla (\nabla \cdot \vec{h}) - \nabla^{2} \vec{h} \\&
	\vec{c} \times (\vec{a} \times \vec{b}) = (\vec{c} \cdot \vec{b}) \vec{a} - (\vec{c} \cdot \vec{a}) \vec{b} \\&
	\vec{a} \cdot (\vec{b} \times \vec{c}) = \vec{b} \cdot (\vec{c} \times \vec{a}) = \vec{c} \cdot (\vec{a} \times \vec{a})
\end{align*}

\paragraph{无旋/无源场}
\begin{equation}
    \nabla\times \vec{F}=0\implies \exists \varphi \,s.t. \vec{F}=\nabla\varphi
\end{equation}
\begin{equation}
    \nabla\cdot \vec{F}=0\implies \exists \vec{C} \,s.t. \vec{F}=\nabla\times\vec{C}
\end{equation}

\paragraph{积分}
\begin{equation}
    flux=\int_{\partial\Omega} \vec{F}\cdot\mathrm{d}\vec{S}
    =\int_{\Omega} \nabla\cdot\vec{F}\mathrm{d}V
\end{equation}
\begin{align*}
    circulation=\int_{\partial\Sigma} \vec{F}\cdot\mathrm{d}\vec{l}
    =\int_{\Sigma} (\nabla\times\vec{F})\cdot\mathrm{d}\vec{S}\\= \int_S \sum (\frac{\partial f_y}{\partial z}-\frac{\partial f_z}{\partial y})dydz
\end{align*}

\subsection{Maxwell equations}
\begin{align}
    \nabla\cdot\vec{E} &= \frac{\rho}{\varepsilon_0} \\
    \nabla \times\vec{E} &= -\frac{\partial \vec{B}}{\partial t} \\
    \nabla\times\vec{B} &= 
    \frac{1}{c^2}\frac{\partial\vec{E}}{\partial t}+\frac{\vec{j}}{c^2\varepsilon_0}\\
    \nabla\cdot \vec{B} &= 0
\end{align}

\section{静电学}
\subsection{basics}

库仑定律
\begin{equation}
    \vec{F}_1 = \frac{1}{4\pi\varepsilon_0}\frac{q_1q_2}{r_{12}^2}\hat{r}_{12}
\end{equation}

电场
\begin{equation}
    \vec{E}=\frac{1}{4\pi\varepsilon_0} \frac{q}{r^2} \hat{r}
\end{equation}

叠加原理
$$
\vec{E} = \frac{1}{4\pi\varepsilon_0} \int_V \frac{\rho \, dV_2}{r^2} \hat{r}
$$
$$
E_x = \frac{1}{4\pi\varepsilon_{0}} \int_V{\frac{(\Delta x)\rho\, dV_2}{r^3}}
$$

电势
\begin{equation}
    \vec{E}=-\nabla\varphi
\end{equation}
$$
\Delta \phi = - \int_{\gamma} \vec E\cdot d\vec s\quad\text{path free}
$$
Gauss定理:闭合面$S$, 有 
\begin{equation}
    \int_S \vec{E}\cdot\mathrm{d}\vec{S} = \frac{Q_{\text{内部}}}{\varepsilon_0}
\end{equation}
\begin{equation}
    \nabla\cdot\vec{E}=\frac{\rho}{\varepsilon_0}
\end{equation}

examples. 
均匀带电球体的场(径向)
\begin{equation}
    E(r)=\frac{1}{4\pi\varepsilon_0}\begin{cases}
        \frac{Q}{r^2}, & r>R_0 \\
        \frac{Qr}{R_0^3}, & r<R_0 
    \end{cases}
\end{equation}

均匀带电直线
\begin{equation}
    E=\frac{\lambda}{2\pi\varepsilon_0 r}
\end{equation}

均匀带电平面
\begin{equation}
    E=\frac{\sigma}{2\varepsilon_0}
\end{equation}
Circle:
$$E(x) = \frac{Q}{4\pi\varepsilon_{0}} \left(\frac{L}{{( r^{2} + L^ { 2}) } ^{\!3/2 }}\right)\,\text{(考虑分量)}$$
Annulus:
$$E_{x} = \frac{\sigma L}{2\varepsilon_{0}}(\frac{1}{\sqrt{{{R}_{1}}^{2}+L^ {2}}} - \frac{1}{\sqrt{{{R}_{2}}^{2} + {{L}} ^{2}}})$$

\subsection{导体}

静电平衡状态下,内部$\rho=0,\vec{E}=0$, 全导体电势相同.
$$
\vec n\cdot \vec E = \frac{\sigma}{\epsilon_0}\quad \vec n\times \vec E = 0
$$
导体空腔:内部无电荷时,内表面无电荷


\subsection{electric dipole}

$+q,-q$电荷,间隔$d$. 定义$p=qd$, $\vec{p}$ 为$-q$指向$+q$的向量,大小为$p$. 
以两者中点为原点,有
\begin{equation}
    \varphi(\vec{r})=\frac{1}{4\pi\varepsilon_0} \frac{\vec{p}\cdot\vec{r}}{r^3}
    =-\frac{1}{4\pi\varepsilon_0}\vec{p}\cdot \nabla\left(\frac{1}{r}\right)
\end{equation}
\begin{equation}
    \vec{E}(\vec{r})=\frac{1}{4\pi\varepsilon_0}\left(
        \frac{3(\vec{p}\cdot\vec{r})\vec{r}}{r^5}-\frac{\vec{p}}{r^3}
    \right)
\end{equation}


\subsection{uniqueness theorem}

唯一性定理:
闭合区域$V$, 已知电荷分布$\rho(\vec{x})$, 若在$S=\partial V$上以下两者之一成立:
\begin{enumerate}
    \item $\varphi|_S$ 给定 
    \item $\frac{\partial\varphi}{\partial \hat{n}}\big|_S$ 给定
    \item 导体所带电荷量确定
\end{enumerate}
则$V$内电场分布唯一确定。 

证明主要利用$\nabla^2\varphi=-\frac{\rho}{\varepsilon_0}$以及恒等式
\begin{equation}
    \int_S u\nabla u\cdot\mathrm{d}\vec{S} = \int_V \nabla\cdot(u\nabla u)\mathrm{d}V
    =\int_V(u\nabla^2u+|\nabla u|^2)\mathrm{d}V.
\end{equation}

唯一性定理与导体:
闭合区域内有一些导体,若以上条件均满足且每个导体电荷量或者电势之一确定,则电场分布唯一。


\subsection{image charge} 

image charge不在希望求电场的空间中,且引入后导体边界面电势不变,
从而可以忽视导体,认为整个空间是Uniform的。
如此引入image charge后,导体外电场不变。

导体平面:对称位置放等量反号电荷

导体球:球外距离球心$d$ 处有电荷$q$, 
则在连线上距离球心$\frac{R^2}{d}$ 处放一个$q'=-\frac{R}{d}q$的电荷,
这两个电荷在球面上每一点产生电势为0.
后可以根据球的电势或者带电量来在球心放置适量电荷。

point charge inside a spherical cavity
- 直接就是 $q$ and $q'$, 无论外导体的带电量,因为内壁永远只产生$-q$的电荷量


\subsection{capacitor}

两极板带电$Q,-Q$, 面积$A$, 距离$d$,电荷全部集中在两个内表面(考虑两个无线平面)
\begin{equation}
    E=\frac{\sigma}{\varepsilon_0}=\frac{Q}{Ad}
\end{equation}
\begin{equation}
    V=Ed=\frac{Qd}{A\varepsilon_0}, \,C=\frac{Q}{V}=\frac{A\varepsilon_0}{d}
\end{equation}



\subsection{complex functions}

$f:\mathbb{C}\to\mathbb{C}$, analytic(解析/可导)。
$f(x+yi)=U(x,y)+iV(x,y)$,$U,V$均可作为电势函数。
取$V$为电势,$U$的差为路径上电场的flux,故$U$的等值线为电场方向
$$
E = -\nabla V =  - \frac{\partial V}{\partial x}-i \frac{\partial V}{\partial y}  = - \frac{\partial V}{\partial x}-i \frac{\partial U}{\partial x} = (-i)\bar f'(z)
$$


\subsection{examples}
\paragraph{plasma oscillation} 
等离子体,几乎电中性,由ions(正电)+electrons(负电)构成。
受到扰动时,电子被推离原位,形成局部正电局部负电(正电ions质量大,受影响小)。
产生恢复电场,如此反复形成oscillation. 
电子数密度$n_0$, plasma frequency 
\begin{equation}
    \omega_p=\sqrt{\frac{q_e^2n_0}{m_e\varepsilon_0}}
\end{equation}

\paragraph{Colloidal particles}

$$\begin{aligned}
\rho &= q_{e} \left(n_{+} - n_{-}\right) \\
     &= q_{e} n_{0} \left(e^{-q_{e} \varphi(x) / kT} - e^{q_{e} \varphi(x) / kT} \right) \\
     &= -2 n_{0} q_{e}^{2} \varphi(x) / kT
\end{aligned}$$
$$\frac{d^2\phi}{dx^2}=\frac{2n_0q_e^2}{\varepsilon_0kT}\phi(x)$$
$$
\phi = \frac{D\sigma}{\epsilon_0}e^{x/D},D=\sqrt{\frac{\epsilon_0kT}{2n_0q_e^2}}
$$

if $n_0 \uparrow$ then $D\downarrow$, $\phi \downarrow$ 静电(排斥)长程力变为短程力,范德华力(吸引)占据主导,产生沉淀

\paragraph{electric field of a grid}

足够远的地方形成均匀电场

$\varphi_n = F_n(z)\cos\frac{2\pi nx}{a} \quad n=1, 2, 3, \ldots$,$\varphi = \sum \varphi_n$
$$
\varphi(x, z) = \underbrace{C - E_0 z}_{\text{均匀项}} + \underbrace{\sum_{n=1}^{\infty} A_n e^{-n\pi z / a} \cos\left(\frac{2n\pi x}{a}\right)}_{\text{衰减项}}
$$


\subsection{静电能}

两个电荷的electrostatic energy: work to bring them together (from infinity) 
\begin{equation}
    U=\frac{q_1q_2}{4\pi\varepsilon_0r_{12}}
\end{equation}

electrostatc energy for a charge system: 
\begin{equation}
    U=\frac{1}{2}\sum_i q_i\varphi_i
\end{equation}
$\varphi_i$: 除$q_i$外,其余charge在$q_i$处产生的电势(规定无穷远处电势为0) 

连续情形: 
\begin{equation}
    U=\frac{1}{2}\int\varphi\mathrm{d}q
\end{equation}

同时有导体和点电荷时的静电能:(点电荷自能不考虑)
\begin{equation}
    U=\frac{1}{2}\left(\sum_i \varphi_iQ_i +\sum_j \varphi_j'q_j\right)
\end{equation}
导体电荷量$Q_i$, $\varphi_i$为该导体在此体系下具有的电势。
点电荷$q_j$, $\varphi_j'$为$q_j$之外的带电体、电荷在此处产生的电势。

电场与电能
\begin{equation}
    U=\frac{\varepsilon_0}{2}\int_\Omega E^2\mathrm{d}V
\end{equation}
$$
U = U_{self}+U_{inter}
$$

自能:
$$
U_{ball} = \frac{3}{5}\frac{Q^2}{4\pi\epsilon_0R},U_{sphere} = \frac{Q^2}{8\pi\epsilon_0R}
$$

电容:
$$U=\frac{1}{2}CV^2$$
\subsection{dielectric \& polarization}
极化率越高,介质内部电场越小,导体有无穷介电常数。注意$P,D,\sigma$有相同的量纲。
\begin{align*}
&\rho_{\text{pol}} = -\nabla \cdot \mathbf{P},\sigma_{pol} =  \vec P \cdot \vec n\\
&P = (\epsilon_r-1)\epsilon_0 E, \quad D = \epsilon_r \epsilon_0 E \\
&\nabla \cdot D = \nabla \cdot (\epsilon_r \varepsilon_0 E) = \rho_{\text{free}} \\
&\nabla \cdot E = \frac{\rho_{\text{free}} + \rho_{\text{pol}}}{\epsilon_0}
\\&E_{1\tau} = E_{2\tau},E_{1n}-E_{2n} = \frac{\sigma_f+\sigma_p}{\epsilon_0},D_{1n}-D_{2n} = \sigma_f
\end{align*}
唯一性定理要求:
$\nabla^2\phi = -\frac{q}{\epsilon_i}$,满足电势的边界条件,满足介质的边界条件




\end{document}